\chapter*{Úvod}\label{chap:uvod}

Ako technológia napreduje, vznikajú stále nové hardvérové a softvérové možnosti v oblasti informatiky. Posledné roky zažili mobilné telefóny obrovský rast. Trend vyrábať čo najmenšie mobilné telefóny dávno skončil a prišli na scénu takzvané múdre telefóny s desať a viac centimetrovými uhlopriečkami obrazoviek a hardvérovým vybavením za ktoré by sa nemusel hanbiť ani prenosný osobný počítač. Tým sa otvorili dvere pre počítačovú grafiku. Vznikla grafická knižnica pre vstavané systémy OpenGL ES \cite{gro} ktorá slúži na komunikáciu s grafickým hardvérom napríklad aj múdrych telefónov. Na tento podnet vznikli aj grafické štruktúry ako jPCT-AE \cite{ols} ktoré slúžia na lepší prístup k metódam knižnice OpenGL ES. Pri svojej práci používam práve spomínanú štruktúru jPCT-AE. Táto štruktúra pracuje s grafickou knižnicou OpenGL ES verziou 1.X aj 2.X a ponúka mi základné metódy využívané pri 3D renderingu v reálnom čase. Pre implementáciu zložitejšieho osvetlenia povrchu 3D objektu ako tieňovanie, lesk, zrkadlové odrazy a ďalšie je nutné použiť knižnicu OpenGL ES 2.X ktorá podporuje vytváranie \textcolor{red}{tieňovačov}. Pomocou týchto \textcolor{red}{tieňovačov} by som chcel dodať 3D objektu v mojej aplikácií čo najrealistickejší vzhľad. Samozrejme musia byť grafické výpočty optimalizované podľa danej hardvérovej zostavy aby ich bola schopná spracovať v reálnom čase. Operačný systém Android pre ktorý je táto aplikácia určená využíva jazyk Java taktiež ako štruktúra jPCT-AE. Takže prirodzene aj ja budem programovať v jazyku Java. Ako vývojové prostredie som si vybral Eclipse \cite{fou} s rozšírením ADT (Android vývojové nástroje) \cite{pro}.
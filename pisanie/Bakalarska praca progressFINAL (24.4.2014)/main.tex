\documentclass[12pt,a4paper,oneside]{bachelor} %{book} mi nechcelo kompilovat kvoli slovencine nejakym zazrakom

\usepackage[slovak]{babel}
\usepackage{amsmath}
\usepackage{amsfonts}
\usepackage{amssymb}
\usepackage[utf8]{inputenc}
\usepackage{graphicx}
\usepackage{times}
\usepackage{listings}
\usepackage{color}
\usepackage{float}

\usepackage{epsfig}
\usepackage{epstopdf}
\usepackage[chapter]{algorithm}
\usepackage{algorithmic}
\usepackage{multirow}
\usepackage{url}
\usepackage{parskip}

\setlength{\parindent}{15pt}

\renewcommand\baselinestretch{1.5}

% konstanty
\def\mftitle{Herný editor vzhľadu karosérii áut}
\def\mfthesistype{Bakalárska práca}
\def\mfauthor{Adam Dudík}
\def\mfadvisor{Prof. RNDr. Roman Ďurikovič, PhD.}
\def\mfplacedate{Bratislava, 2014}

\ifx\pdfoutput\undefined\relax\else\pdfinfo{ /Title (\mftitle) /Author (\mfauthor) /Creator (PDFLaTeX) } \fi

%koniec nastaveni a tu uz zacina obsah
\begin{document}

%\frontmatter

\thispagestyle{empty}

\noindent
\begin{minipage}{\textwidth}
\begin{center}
\uppercase{Univerzita Komenského v Bratislave \\
Fakulta matematiky, fyziky a informatiky}
\end{center}
\end{minipage}

\vfill
\begin{center}
\begin{minipage}{0.8\textwidth}
\centerline{\textbf{\LARGE\MakeUppercase{\mftitle}}}
\smallskip
\centerline{\mfthesistype}
\end{minipage}
\end{center}
\vfill
2014 \hfill
\mfauthor
\eject 
% koniec obalu

\thispagestyle{empty}

\begin{minipage}{\textwidth}
\begin{center}
\uppercase{Univerzita Komenského v Bratislave \\
Fakulta matematiky, fyziky a informatiky}
\end{center}
\end{minipage}

\vfill
\begin{center}
\begin{minipage}{0.8\textwidth}
\centerline{\textbf{\MakeUppercase{\mftitle}}}
\smallskip
\centerline{\mfthesistype}
\end{minipage}
\end{center}
\vfill
\begin{tabular}{l l}
Registračné číslo: &  \\
Študijný program: & aplikovaná informatika\\
Študijný odbor: & 2511 aplikovaná informatika\\
Školiace pracovisko: & Katedra Informatiky\\
Školiteľ: & \mfadvisor
\end{tabular}
\vfill
\mfplacedate \hfill
\mfauthor
\eject 

\thispagestyle{empty}

{~}\vspace{12cm}

\noindent
\begin{minipage}{0.25\textwidth}~\end{minipage}
\begin{minipage}{0.75\textwidth}
Čestne prehlasujem, že túto bakalársku prácu som vypracoval samostatne s použitím uvedených zdrojov a radami môjho školiteľa.
\newline \newline
\end{minipage}
\vfill
~ \hfill {\hbox to 6cm{\dotfill}} \\
\mfplacedate \hfill Adam Dudík
\vfill\eject 
% koniec prehlasenia

\chapter*{Poďakovanie}\label{chap:dakovanie}
Chcel by som poďakovať vedúcemu mojej bakalárskej práce Prof. RNDr. Romanovi Ďurikovičovi, PhD. za rady a pripomienky pri vypracovávaní tejto práce.
\vfill\eject 
% koniec podakovania

\chapter*{Abstrakt}\label{chap:abstrakt_sk}
Naštudovať si knižnice OpenGL ES pre prostredie Android, preštudovať metódy renderingu v reálnom čase, pochopyť princíp textúrovania a modelovania lesku. Implementovať jednoduchý renderer 3D modelu auta s užívateľským výberom lakov a textúr.

~\\
Kľúčové slová: OpenGL ES, Android, rendering v reálnom čase, textúrovanie, lesk, 3D model
\vfill\eject 

\chapter*{Abstract}\label{chap:abstrakt_en}
Study OpenGL ES libraries Android environment, study methods of real-time rendering, understand principles of texturing and gloss modeling. Implement basic renderer of a car 3D model with user's choice of paint and textures.\textbf{
}
~\\
Keywords: OpenGL ES, Android, real-time rendering, texturing, gloss, 3D model
\vfill\eject 
% koniec abstraktov

\input 00predhovor.tex

\tableofcontents

\listoffigures

%\mainmatter

\input 01uvod.tex
\input 02prehlad.tex
\input 03ciel.tex

%\backmatter

\nocite{*}
\bibliographystyle{alpha}
\bibliography{references}


\end{document}
